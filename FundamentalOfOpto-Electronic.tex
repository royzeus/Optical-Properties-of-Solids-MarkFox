% !TEX TS-program = xelatex
% !TEX encoding = UTF-8

\documentclass[12pt]{book} % use larger type; default would be 10pt

\usepackage{fontspec} % Font selection for XeLaTeX; see fontspec.pdf for documentation
\defaultfontfeatures{Mapping=tex-text} % to support TeX conventions like ``---''
\usepackage{xunicode} % Unicode support for LaTeX character names (accents, European chars, etc)
\usepackage{xltxtra} % Extra customizations for XeLaTeX

\setmainfont{Adobe Song Std} % set the main body font (\textrm), assumes Charis SIL is installed
%\setsansfont{Deja Vu Sans}
%\setmonofont{Deja Vu Mono}

% other LaTeX packages.....
\usepackage{geometry} % See geometry.pdf to learn the layout options. There are lots.
\geometry{a4paper} % or letterpaper (US) or a5paper or....
%\usepackage[parfill]{parskip} % Activate to begin paragraphs with an empty line rather than an indent

\usepackage{graphicx} % support the \includegraphics command and options

\title{Fundamental of Opto-Electronics}
\author{Jiaguang Han}

\begin{document}

\maketitle

\chapter{Introduction}

\section{Maxwell}
\begin{eqnarray}
% \nonumber to remove numbering (before each equation)
  \nabla\times\mathbf{E} &=& -\frac{\partial\mathbf{B}}{\partial t} \\
  \nabla\times\mathbf{H} &=& \mathbf{J}+\frac{\partial\mathbf{D}}{\partial t} \\
  \nabla\cdot\mathbf{B} &=& 0 \\
  \nabla\cdot\mathbf{D} &=& \rho
\end{eqnarray}

\begin{eqnarray}
% \nonumber to remove numbering (before each equation)
  \oint_{loop}\mathbf{E}\cdot d\mathbf{l} &=& -\frac{\partial}{\partial t}\int_{area}\mathbf{B}\cdot d\mathbf{S} \\
  \oint\mathbf{H}\cdot d\mathbf{l} &=& \int\mathbf{J}\cdot d\mathbf{S}+\frac{\partial}{\partial t}\int_{area}\mathbf{D}\cdot d\mathbf{S} \\
  \oint_{surf}\mathbf{B}\cdot d\mathbf{S} &=& 0 \\
  \oint_{surf}\mathbf{D}\cdot d\mathbf{S} &=& Q_{enclosed}
\end{eqnarray}

\begin{equation}\label{equa:lorrents}
  F=q(\mathbf{e}+\mathbf{v}\times\mathbf{B})
\end{equation}

$\mathbf{E}$:electric field strength;$V/M$

$\mathbf{D}$:electric flux density; $C/m^2$

$\mathbf{H}$:magnetic field strength; $A/m$

$\mathbf{B}$:magnetic flux density; $Webers/m^2$

$\mathbf{J}$:electric current density; $A/m^2$

$I_0$;$I_{incident},I_{reflection},I_{transmission},I_{absorption},I_{lumin},I_{scattering},I_{interference}$

$\displaystyle R=\frac{|E_R|^2}{|E_0|^2}$:reflectivity, or reflectance\quad$\displaystyle T=\frac{|E_T|^2}{|E_0|^2}$:transmitivity, or transmitance

$\displaystyle r=\frac{|E_R|}{|E_0|}$:reflection coefficiency\quad$\displaystyle t=\frac{|E_T|}{|E_0|}$:transmit coefficiency

$\displaystyle \lambda=c/f=2\pi c/\omega; \frac{v}{c}=\frac{\lambda_1 f_1}{\lambda_0 f_0}=\frac{1}{n_1}$, so $f_0=f_1$

When $l\gg l_c$,
\begin{equation}\label{equa:trans}
  T=\frac{(1-R_1)(1-R_2)e^{-\alpha l}}{1-R_1R_2e^{-2\alpha l}}={(1-R_1)(1-R_2)e^{-\alpha l}}\cdot{\sum_0^\infty(R_1R_2e^{-2\alpha l})^i}
\end{equation}

$I=\langle s \rangle\propto\langle \mathbf{E}\times\mathbf{H}\rangle=\frac{c}{4\pi}\sqrt{\frac{\varepsilon}{\mu}}\langle E^2 \rangle$

 for $\mathbf{E}(\mathbf{r},t)=\frac{1}{2}[A(\mathbf{r})e^{-i\omega t}+A^*(\mathbf{r})e^{i\omega t}]$,$E=E_1+E_2$
 \begin{equation}
 I=\langle E^2 \rangle=\langle E_1^2\rangle+\langle E_2^2\rangle+2\cdot\langle E_1\cdot E_2\rangle
 \end{equation}

 if $\omega_1=\omega_2$,
 $$\langle E_1\cdot E_2\rangle=\frac{1}{4}\langle A_1A_2^*+A_1^*A_2\rangle$$

 if $A_1=a_1e^{i\phi_1},A_2=a_2e^{i\phi_2}$
 $$\langle E_1 E_2\rangle=\frac{1}{4}[a_1a_2cos(\phi_1-\phi_2)+a_1a_2cos(\phi_2-\phi_1)]$$
 that is to say ,for $\delta=\phi_1-\phi_2={2\pi}/{\lambda}L_c$, $\displaystyle L_c=\frac{\lambda\cdot\delta}{2\pi}$



Absorption coefficiency:$\alpha$, according to \textbf{Lamb-Beer Law}:$I=I_0e^{-\alpha l}$.
Absorb ratio: $\displaystyle A=|\frac{I_\alpha}{I_0}|=|\frac{E_\alpha}{E_0}|$

\section{Polarization}

$$\mathbf{H}=(-\mathbf{x}\frac{b}{\eta}e^{j\phi_b}e^{-jkz}+\mathbf{y}\frac{a}{\eta}e^{j\phi_a}e^{-jkz})e^{j\omega t}$$

for $z=0$

$$\mathbf{H}=(-\mathbf{x}\frac{b}{\eta}e^{j\phi_b}+\mathbf{y}\frac{a}{\eta}e^{j\phi_a})e^{j\omega t}$$

\begin{eqnarray}
  \mathbf{E} &=& (\mathbf{x}ae^{j\phi_a}+\mathbf{y}be^{j\phi_b})e^{j\omega t}  \\
  real(\mathbf{E}) &=& \mathbf{x}a\cos(\omega t+\phi_a)+\mathbf{y}b\cos(\omega t+\phi_b)
\end{eqnarray}

for $a=b, \phi_a=\phi_b$, linear-polarization

for$a=b,\phi_a-\phi_b=\pi/2$, circle-polarization

\section{Material}

with a $\epsilon$ and $\mu$, the material is called \textbf{isotropic};

with 9$\epsilon$ and 9$\mu$, \textbf{anisotropic};

with 18$\epsilon$ and 18$\mu$, \textbf{bianisotropic};

with $\epsilon$, $\mu$, $\chi$, $\kappa$, \textbf{baisotropic};

homogeneous: independent on $\mathbf{r}$. $\epsilon(\mathbf{r})=\epsilon_0$;

linear material: $\epsilon{\mathbf{E}}=\epsilon_0=\epsilon(\mathbf{r},t)$



\chapter{Classical Propagation}
\chapter{Interband absorption}
\chapter{Excitons}
\chapter{Luminescence}
\chapter{Free Electrons}
\chapter{Phonons}


\end{document}
